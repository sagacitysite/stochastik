\documentclass{article}
\usepackage[a4paper,left=2cm,right=2cm,top=2cm,bottom=2.5cm]{geometry} 
\usepackage[utf8]{inputenc}
\usepackage[ngerman]{babel}
\usepackage{amsmath}
\usepackage{amssymb}
\usepackage{bbm}
\usepackage{BOONDOX-cal}

\setlength\parindent{0pt}
\setlength{\parskip}{8pt}
\renewcommand{\arraystretch}{1.25}

\begin{document}

\section{Maßtheorie}

\subsubsection*{Algebra und Maß}

\begin{itemize}
\item Eine \emph{Algebra} $\mathcal{A}$ auf $A$ ist ein Mengensystem, das $A$ enthält und abgeschlossen/stabil ist bzgl. paarweiser Vereinigung und Komplementbildung.
\item Ein \emph{Prämaß} auf $\mathcal{A}$ ist eine Funktion $\mu_A : \mathcal{A} \to \mathbb{\bar{R}_+}$, für die $\mu(\emptyset) = 0$ und $\sigma$-Additivität gilt.
\item Eine $\sigma$-\emph{Algebra} $\mathcal{B}$ auf $B$ ist ein Mengensystem, das $B$ entäkt und abgeschlossen/stabil ist bzgl. abzählbar unendlicher Vereinigung und Komplementbildung.
\item Ein \emph{Maß} auf $\mathcal{B}$ ist eine Funktion $\mu_B : \mathcal{B} \to \mathbb{\bar{R}_+}$, für die $\mu(\emptyset) = 0$ und $\sigma$-Additivität gilt.
\end{itemize}

\subsubsection*{Eigenschaften}

Sei $\mathcal{F}$ $\sigma$-Algebra auf $\Omega$, $A \in \mathcal{F}$, $(A_n)$ Folge von Teilmengen von $\Omega$ (d.h. $A_n \in \mathcal{F}$) und $\mu$ Maß.

\begin{tabular}{ll}
Messbarer Raum & $(\Omega, \mathcal{F})$ \\
$\sigma$-Additivität & $\mu(\bigcup_{n=1}^\infty A_n) = \sum_{n=1}^\infty \mu(A_n)$ \\
$\sigma$-Stetigkeit & $\mu(A) = \lim_{n \to \infty} \mu(A_n)$, wobei $A_n \nearrow A$, d.h. $A_n \subseteq A_{n+1}$ und $\bigcup_n A_n = A$ \\
& \\
Endliches Maß & $\mu(\Omega) < \infty$ \\
$\sigma$-endliches Maß & $\Omega = \bigcup_{n=1}^\infty A_n$ und $\mu(A_n) < \infty$ \\
W-Maß & $\mu(\Omega) = 1$ und $\sigma$-Additivität für paarweise disjunkte $A_n \in \mathcal{F}$ (Kolmogorov Axiome) \\
\end{tabular}

\subsubsection*{Messbare Abbildungen}

Seien $(\Omega, \mathcal{F})$ und $(S, \mathcal{S})$ messbare Räume und $f: \Omega \to \mathcal{F}$ eine Abbildung.

\begin{tabular}{ll}
$(\mathcal{F},\mathcal{S})$-messbare Abbildung & $f^{-1}(B) \in \mathcal{F}$ für alle $B \in \mathcal{S}$ \\
Borel-messbare Abbildung & $(S,\mathcal{S}) = (\mathbb{R},\mathcal{B}(\mathbb{R}))$ \\
\end{tabular}

Alle stetigen Funktionen sind Borel-messbar.

\section{Wahrscheinlichkeitsräume}

\subsection{Grundlagen}

\begin{tabular}{ll}
$\Omega$ & \emph{Ergebnismenge}, wobei $\omega_i \in \Omega$ \emph{Ergebnis} (Grundmenge)\\
$\mathcal{F}$ & \emph{Ereignissystem}, wobei $A \in \mathcal{F}$ \emph{Ereignis} ($\sigma$-Algebra über der Grundmenge $\Omega$, d.h. $\mathcal{F} = \sigma(\Omega)$) \\
$(\Omega, \mathcal{F})$ & \emph{Ereignisraum} (Messraum, messbarer Raum) \\
$\mathbb{P}$ & \emph{Wahrscheinlichkeitsmaß} (Maß) \\
$(\Omega, \mathcal{F}, \mathbb{P})$ & \emph{Wahrscheinlichkeitsraum} (Maßraum) \\
& \\
$X$ & $(S,\mathcal{S})$-wertige \emph{Zufallsvariable}, wobei $X: \Omega \to S$ (Abbidung) \\
$(S,\mathcal{S})$ & Messraum mit Grundmenge $S$ und $\sigma$-Algebra $\mathcal{S} = \sigma(S)$\\
$\mathbb{P^X}$ & \emph{Wahrscheinlichkeitsverteilung} (von $\mathbb{P}$ induziertes Maß) \\
$(S,\mathcal{S},\mathbb{P}^X)$ & Maßraum, wobei das enthaltene Maß der Wahrscheinlichkeitsverteilung entspricht \\
\end{tabular}

Für eine Zufallsvariable $X$ und die $\sigma$-Algebren $\mathcal{F}$ und $\mathcal{S}$ gilt:

\begin{tabular}{llcl}
Für & $A \in \mathcal{F}$ & gilt & $X(A) \in \mathcal{S}$ \\
Für & $B \in \mathcal{S}$ & gilt & $X^{-1}(B) \in \mathcal{F}$, wobei $X^{-1}(B) = \{\omega \in \Omega : X(\omega) \in B\}$ \\
\end{tabular}

Für das Wahrscheinlichkitsmaß $\mathbb{P}$ und die Wahrscheinlichkeitsverteilung $\mathbb{P}^X$ gilt:

\begin{tabular}{llcl}
Für & $A \in \mathcal{F}$ & gilt & $\mathbb{P}(A) = \mathbb{P}^X \circ X(A) = \mathbb{P}^X(X(A))$ \\
Für & $B \in \mathcal{S}$ & gilt & $\mathbb{P^X}(B) = \mathbb{P} \circ X^{-1}(B) = \mathbb{P}(X^{-1}(B)) = \mathbb{P}(\{\omega \in \Omega : X(\omega) \in B\}) = \mathbb{P}(\{X \in B\}) \overset{kurz}{=} \mathbb{P}(X \in B)$ \\
\end{tabular}

Zusammengefasst gilt:

$[0,1] \overset{\mathbb{P}}{\longleftarrow} (\Omega, \mathcal{F}, \mathbb{P}) \overset{X^{-1}}{\longleftarrow} (S, \mathcal{S}, \mathbb{P}^X) \overset{\mathbb{P}^X}{\longrightarrow} [0,1]$

\subsection{Wahrscheinlichkeitsverteilung, -dichte und Verteilungsfunktion}

\subsubsection*{Diskreter Wahrscheinlichkeitsraum}

Diskreter Wahrscheinlichkeitsraum, falls $\Omega$ endlich oder abzählbar unendlich.

Sei $\omega \in \Omega$ und $s \in S$, sowie $A \in \mathcal{F}$ und $B \in \mathcal{S}$ dann gilt mit einer \emph{Wahrscheinlickeitsdichtefunktion}, auch kurz Wahrscheinlichkeitsdichte $p$ bzw. $p^X$ (entspricht \emph{Zähldichte} im diskreten Fall):

\begin{tabular}{ll}
$\mathbb{P}(\{\omega\}) = p(\omega)$ & Zähldichte $p(\omega)$ bestimmt $\mathbb{P}$ eindeutig bzgl. diskretem Wahrscheinlichkeitsraum\\
$\mathbb{P}(A) = \sum_{\omega \in \Omega} p(\omega) \mathbbm{1}_A(\omega)$ & Wahrscheinlichkeitsmaß $\mathbb{P}$ mit Zähldichte $p(\omega)$ und Zählmaß $\mathbbm{1}_A(\omega)$ \\
$\mathbb{P}^X(\{s\}) = p^X(s)$ & Zähldichte $p^X(s)$ bestimmt $\mathbb{P}^X$ eindeutig bzgl. diskret verteilter $(S,\mathcal{S})$-wertiger ZV\\
$\mathbb{P}^X(B) = \sum_{s \in S} p^X(s) \mathbbm{1}_A(s)$ & Wahrscheinlichkeitsverteilung $\mathbb{P}^X$ mit Zähldichte $p^X(s)$ und Zählmaß $\mathbbm{1}_A(s)$ \\
\end{tabular}

dabei gilt $\mathbb{P}(\Omega) = \sum_{\omega \in \Omega} p(\omega) = 1$ und $\mathbb{P}^X(S) = \sum_{s \in S} p(s) = 1$

Für eine Zufallsvariable $X$ gilt für die \emph{Verteilungsfunktion} $F$:

$F(x) = \mathbb{P}(\{ X \le x \}) = \sum_{k=1}^{\lfloor x \rfloor} \mathbb{P}(\{ X = k \})$

\subsubsection*{Stetiger Wahrscheinlichkeitsraum}

Stetiger Wahrscheinlichkeitsraum, falls $\Omega$ überabzählbar.

$A \in \mathcal{F}$ und $B \in \mathcal{S}$ dann gilt mit einer \emph{Wahrscheinlickeitsdichtefunktion} $p$ bzw. $p^X$:

\begin{tabular}{ll}
$\mathbb{P}(A) = \int_A p(\omega)\, \lambda(d\omega)$ & Wahrscheinlichkeitsmaß $\mathbb{P}$ mit W-Dichtefkt. $p(\omega)$ und Lebesque-Maß $\lambda$ \\
$\mathbb{P}^X(B) = \int_{B} p^X(s)\, \lambda(ds)$ & Wahrscheinlichkeitsverteilung $\mathbb{P}^X$ mit W-Dichtefkt. $p^X(s)$ und Lebesque-Maß $\lambda$ \\
\end{tabular}

dabei gilt $\mathbb{P}(\Omega) = \int_{\Omega} p(\omega) \lambda(d\omega) = 1$ und $\mathbb{P}^X(S) = \int_{S} p(s) \lambda(ds) = 1$

\subsubsection*{Stetiger reeller Wahrscheinlichkeitsraum}

Spezialfall ($\Omega,\mathcal{F}) = (\mathbb{R}, \mathcal{B}(\mathbb{R}))$:

$(a_1,a_2] \in \mathcal{F}$ und $(b_1,b_2] \in \mathcal{S}$ dann gilt mit einer \emph{Wahrscheinlickeitsdichtefunktion} $f$ bzw. $f^X$:

\begin{tabular}{ll}
$\mathbb{P}((a_1,a_2]) = \int_{a_1}^{a_2} f(x)\, dx$ & Wahrscheinlichkeitsmaß $\mathbb{P}$ mit W-Dichtefkt. $f(x)$ \\
$\mathbb{P}^X((b_1,b_2]) = \int_{b_1}^{b_2} f^X(x)\, dx$ & Wahrscheinlichkeitsverteilung $\mathbb{P}^X$ mit W-Dichtefkt. $f^X(x)$ \\
\end{tabular}

Für eine Zufallsvariable $X \in \mathbb{R}$ gilt für die \emph{Verteilungsfunktion} $F$:

$F(x) = \mathbb{P}(\{X \le x\}) = \mathbb{P}^X((-\infty,x]) = \int_{(-\infty,x]} d\mathbb{P}^X = \int_{(-\infty,x]} d(\mathbb{P}\circ X^{-1}) = \int_{(-\infty,x]} f^X(x)\,\lambda(dx) = \int_{-\infty}^x f^X(x)\, dx$

\subsection{Häufige W-Dichten, W-Verteilungen und Verteilungsfunktionen}

\subsubsection*{Diskrete Gleichverteilung}

\begin{tabular}{ll}
Wahrscheinlichkeitsmaß & $\mathbb{P}(A) = \frac{|A|}{|S|}$ \\
Wahrscheinlichkeitsdichte & $p_{U(S)}(k) = \mathbb{P}(\{X = k\}) = \frac{1}{|S|}$ \\
Verteilungsfunktion & $F_{U(S)}(x) = \mathbb{P}(\{X \le x\}) = \frac{|\{i:\, k_i \le x\}|}{|S|}$ \\
\end{tabular}

\subsubsection*{Bernoulli-Schema}

\begin{tabular}{ll}
Wahrscheinlichkeitsdichte & $p_{B(p)}(k) = \mathbb{P}(\{X = k\}) = p^k\,(1-p)^{n-k}$ \\
\end{tabular}

\subsubsection*{Binomialverteilung}

\begin{tabular}{ll}
Wahrscheinlichkeitsdichte & $p_{B(n,p)}(k) = \mathbb{P}(\{X = k\}) = \binom{n}{k}\,p^k\,(1-p)^{n-k}$ \\
Verteilungsfunktion & $F_{B(n,p)}(x) = \mathbb{P}(\{X \le x\}) = \sum_{k=0}^{\lfloor x \rfloor} \binom{n}{k}\,p^k\,(1-p)^{n-k}$ \\
\end{tabular}

\subsubsection*{Geometrische Verteilung}

\begin{tabular}{ll}
Wahrscheinlichkeitsdichte & $p_{G(n,p)}(k) = \mathbb{P}(\{X = k\}) = p\,(1-p)^{n-1}$ \\
Verteilungsfunktion & $F_{G(n,p)}(x) = \mathbb{P}(\{X \le x\}) = p\, \sum_{k=0}^{\lfloor x \rfloor} (1-p)^{n-1} \overset{x\;\text{diskret}}{=} 1-(1-p)^n $ \\
\end{tabular}

\subsubsection*{Poissonverteilung}

\begin{tabular}{ll}
Wahrscheinlichkeitsdichte & $p_{P(\lambda)}(k) = \mathbb{P}(\{X = k\}) = \frac{\lambda^k}{k!}\, e^{-\lambda}$ \\
Verteilungsfunktion & $F_{P(\lambda)}(x) = \mathbb{P}(\{X \le x\}) =  e^{-\lambda}\, \sum_{k=0}^{\lfloor x \rfloor} \frac{\lambda^k}{k!}$ \\
\end{tabular}

\subsubsection*{Stetige Gleichverteilung}

\begin{tabular}{ll}
Wahrscheinlichkeitsdichte & $f_{U(A)}(x) = \mathbb{P}(\{X \in A\}) = \frac{1}{\lambda(A)}\, \mathbbm{1}_A(x)$, $A \in \mathcal{B}(\mathbb{R})$ \\
W-Dichte für Intervalle & $f_{U([a,b])}(x) = \frac{1}{b-a}\, \mathbbm{1}_{[a,b]}(x)$, $a < b \in \mathbb{R}$ \\
Verteilungsfunktion & $F_{U([a,b])}(x) = \mathbb{P}(\{X \le x\}) = \frac{x-a}{b-a}\, \mathbbm{1}_{[a,b]}(x) + \mathbbm{1}_{(b,\infty)}(x)$ \\
\end{tabular}

\subsubsection*{Expoentialverteilung}

\begin{tabular}{ll}
Wahrscheinlichkeitsdichte & $f_{E(\lambda)}(x) = \lambda\,e^{-\lambda x}\, \mathbbm{1}_{[0,\infty)}(x)$ \\
Verteilungsfunktion & $F_{E(\lambda)}(x) = \mathbb{P}(\{X \le x\}) = \int_0^{\infty} \lambda\,e^{-\lambda x}\, dx = (1 - e^{-\lambda x})\, \mathbbm{1}_{[0,\infty)}(x)$ \\
\end{tabular}

\subsubsection*{Normalverteilung}

\begin{tabular}{ll}
Wahrscheinlichkeitsdichte & $f_{N(\mu,\sigma)}(x) = \frac{1}{\sqrt{2\pi \sigma^2}}\, \exp\left( -\frac{(x-\mu)^2}{2\sigma^2} \right)$ \\
Verteilungsfunktion & $F_{N(\mu,\sigma)}(x) = \mathbb{P}(\{X \le x\}) = \int_{-\infty}^{\infty} \frac{1}{\sqrt{2\pi \sigma^2}}\, \exp\left( -\frac{(x-\mu)^2}{2\sigma^2} \right) = \Phi(\frac{x-\mu}{\sigma})$ \\
\end{tabular}

\section{Unabhängigkeit}

Seien $(\Omega_1, \mathcal{F_1})$ und $(\Omega_2, \mathcal{F_2})$ Ereignisräume, sowie $(S_1, \mathcal{S_1})$ und $(S_2, \mathcal{S_2})$ messbare Räume, $X_1: \Omega_1 \longrightarrow S_1$ und $X_2: \Omega_2 \longrightarrow S_2$ Zufallsvariablen.

\begin{tabular}{lll}
Ereignisse & $A_1$ und $A_2$ unabhängig, wenn & $\mathbb{P}(A_1 \cap A_2) = \mathbb{P}(A_1) \cdot \mathbb{P}(A_2)$ für $A_1,A_2 \in \mathcal{F_1}$ \\
$\sigma$-Algebren & $\mathcal{F_1}$ und $\mathcal{F_2}$ unabhängig, wenn & $\mathbb{P}(A \cap B) = \mathbb{P}(A) \cdot \mathbb{P}(B)$ für alle $A \in \mathcal{F_1}$, $B \in \mathcal{F_2}$ \\
Zufallsvariablen & $X_1$ und $X_2$ unabhängig, wenn &  $\sigma$-Algebren $X^{-1}_1(\mathcal{S_1})$ und $X^{-1}_2(\mathcal{S_2})$ unabhängig \\
\end{tabular}

\section{Konvergenzen}

\begin{tabular}{lll}
Fast sichere Konvergenz & $X_n \overset{f.s.}{\longrightarrow} X$ & $\mathbb{P}(\{\omega \in \Omega : \lim_{n \to \infty} X_n(\omega) = X(\omega)\}) = 1$ \\
Konvergenz im $p$-ten Mittel & $X_n \overset{L_p}{\longrightarrow} X$ & $\lim_{n \to \infty} \mathbb{E}[|X_n-X|^p] = 0$ \\
Konvergenz im quadr. Mittel & $X_n \overset{L_2}{\longrightarrow} X$ & $\lim_{n \to \infty} \mathbb{E}[(X_n-X)^2] = 0$ (Spezialfall) \\
Stochastische Konvergenz & $X_n \overset{\mathbb{P}}{\longrightarrow} X$ & $\forall\, \epsilon > 0:\, \lim_{n \to \infty} \mathbb{P}(\{|X_n-X| > \epsilon\}) = 0$ \\
Konvergenz in Verteilung & $X_n \overset{d}{\longrightarrow} X$ & $\lim_{n \to \infty} F_{X_n}(x) = F_X(x)$ bzw. $\lim_{n \to \infty} \mathbb{E}[\varphi(X_n)] = \mathbb{E}[\varphi(X)]$ \\
Schwache Konvergenz & $\mathbb{P}^{X^n} \overset{w}{\longrightarrow} \mathbb{P}$ &  $\lim_{n \to \infty} \int_{\mathbb{R}^d} \varphi(x)\, \mathbb{P}_n(dx) = \int_{\mathbb{R}^d} \varphi(x)\, \mathbb{P}(dx)$ \\
\end{tabular}

Stochastische Konvergenz wird auch als Konvergenz in Wahrscheinlichkeit bezeichnet.

\subsubsection*{Beziehungen}

Konvergenz in Verteilung $\Leftrightarrow$ Schwache Konvergenz

Fast sicher $\Rightarrow$ Stochastisch $\Rightarrow$ In Verteilung

Im $p$-ten Mittel $\Rightarrow$ Stochastisch $\Rightarrow$ In Verteilung

Fast sicher $\not\Leftrightarrow$ Im $p$-ten Mittel

\section{Formeln}

\subsubsection*{Vandermondesche Identität}

\[ \sum_{k=0}^n \binom{M}{k} \binom{N}{n-k} = \binom{M+N}{n} \]

\subsubsection*{Binomischer Lehrsatz}

\[ (x+y)^n = \sum_{k=0}^n \binom{n}{k}\, x^n\, y^{n-k} \]

\subsubsection*{Gauß-Integral}

\[ \int_{-\infty}^{\infty} \exp\left( -\frac{x^2}{2} \right)\, dx = \sqrt{2\pi} \]

\subsubsection*{Imaginäres Exponential-Integral}

\[ \int_{-a}^{a} e^{iux}\, dx = \frac{2\, \sin(au)}{u} \]

\subsubsection*{Reihen}

\[ \sum_{n=0}^{\infty} \frac{x^n}{n!} = e^x \qquad \text{(Exponentialreihe)}\]

\[ \sum_{n=0}^{\infty} x^n = \frac{1}{1-x},\; |x|<1 \qquad \text{(Geometrische Reihe)} \]

\subsubsection*{Gamma-Funktion}

\[ \Gamma(x) = \int_0^{\infty} t^{x-1}\,e^{-t}\,dt \]

\[ \text{Es gilt:} \quad \Gamma(\frac{1}{2}) = \sqrt{\pi}, \quad \Gamma(x+1) = x!, \quad \Gamma(x+1) = x\,\Gamma(x), \quad \Gamma(1) = 1 \]

\subsubsection*{Dreiecksungleichung}

\[ |a+b| \le |a| + |b|, \qquad \biggm| \sum_{i=1}^n x_i \biggm| \le \sum_{i=1}^n |x_i|, \qquad \biggm| \int_I f(x)\,dx \biggm| \le \int_I |f(x)| \]

\end{document}
